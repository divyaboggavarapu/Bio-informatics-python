\documentclass{article}
\usepackage[utf8]{inputenc}
\usepackage{minted}
\usepackage{hyperref}
\usepackage{amsmath,amssymb}
\usepackage{graphicx}


\title{\begin{center}CSE 5370: Bioinformatics Homework-2\end{center}}
\author{Divya Boggavarapu(102086719)}
\date{October 2022}

\begin{document}

\maketitle

\subsection*{Collaboration Statement}
I have collaborated with Vijitha Kotapati(1001860730) and Tulasi Sridevi Navuluru(1002010740) to do the programming questions in 2,3 and 4  by understanding the Needleman-Wunsch, Smith-Waterman algorithms. 

\section{Substitution Matrices}

Suppose that transition mutations (A ←→G and T ←→C) are less common
than tranversions (A ←→ T , A ←→ C, G ←→ T , and G ←→ C). Following is the substitution matrix that reflects this.
\\

\begin{matrix}
\hline
&A&G&T&C \\
\hline
A&1&-5&-1&-1\\
G&-5&1&-1&-1\\
T&-1&-1&1&-5\\
C&-1&-1&-5&1\\
\end{matrix}
\\



\section{Global Alignment }
we have conducted global alignment with the Needleman-Wunsch
algorithm and implemented it in a single python file
called \textbf{1002086719\_NW.py} 
\\

The global alignment function in the file will take in
sequence A and sequence B as strings to be aligned (assume that these
strings only contain the chars “acgt”), a gap penalty, and a
substitution matrix and returns a list of tuples representing possible
alignments.

\subsection{An example}

In \textbf{1002086719\_NW.py}, we executed the global alignment function with input strings “GATA”
and “CTAC”, substitution matrix d and gap penalty as parameters which returned the tuples ("GATA", "-CTAC"),(“GATA-”,“C-TAC”).








\section {Local Alignment}
we have conducted local alignment with the Smith-Waterman algorithm and
implemented it in a single python file
called \textbf{1002086719\_SW.py}
           
\\

The local\_alignment function in the file will take in
sequence A and sequence B as the strings to be aligned (assume that these
strings only contain the chars “acgt”), a gap penalty, and a
substitution matrix and returns a list of tuples representing possible
alignments.

\section{A Custom Alignment }
\begin{itemize}
  \item I have taken my first name \textbf{divya}, last name \textbf{boggavarapu} in lowercase and concatenated them to be \textbf{'divyaboggavarapu'}.
  \item I have written code to create a custom substitution dict to substitute all the 26 English alphabets as characters and included code for this problem in the file \textbf{"1002086719\_CUSTOM.py"}
  
  The file \textbf{"1002086719\_S.txt"} will provide the output of my pretty matrix.
  
  \item I ran “local\_alignment” function from Problem 3 with the custom S defined by my name, a gap penalty of -2, my concatenated name (i.e. “divyaboggavarapu”) as the first string, and the pangram “thequickbrown- foxjumpsoverthelazydog” as the second string. Following are the output tuples:
  \\
  
 \textbf{('\_\_\_\_\_\_\_ggavarapu', 'thequicwnfoxjump'), ('\_\_\_\_\_\_\_\_\_\_\_\_\_arapu', 'thequickobvroerthe')}

  
  \item The file \textbf{"1002086719\_D.txt"} will provide the output of my pretty matrix. 
\end{itemize}

%When writing this homework assignment, I referred the following links to learn about NeedleMan-Wunsch algorithm and Smith-Waterman Algorithm.
%https://github.com/CharliesCodes/bioinformatics/blob/main/algorithms/needleman_wunsch.py
%https://github.com/CharliesCodes/bioinformatics/blob/main/algorithms/smith_waterman.py
%https://github.com/haekalyulianto/Needleman_Wunsch/blob/main/Bioinfomatics_Implementation_NeedlemanWunsch_Haekal_17_415900_PA_18169.ipynb
%https://github.com/bott0m/Needleman-Wunsch-Algorithm/blob/master/global_align.py

\section{Difficulty Adjustment}
Please find my feedback below: 
\begin{itemize}
    \item  How long did this assignment take you to complete?
    \begin{itemize}
        \item \textbf{I completed my assignment in 14 hours.}
    \end{itemize}
    \item  If the assignment took you longer than the 10 hours, which parts were overly difficult?
    \begin{itemize}
        \item \textbf{Faced difficulty in understanding the First and Custom alignment questions.}
    \end{itemize}
\end{itemize}

\end{document}
\end{document}